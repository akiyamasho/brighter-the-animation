\section{Introduction}

Japanese Animation, or anime, has been enjoyed as a form of recreation and expression both in Japan and internationally, and is now a 200 billion-yen industry~\cite{animeIndustry200B, animeIndustry200B2}. Despite the its massive impact and growth, there are still major issues that can be observed in the efficiency of its production process and more importantly the work environment~\cite{NHKAnimeIndustry}. Current anime studios, with the exception of some studios such as Kyoto Animation~\cite{kyoAniRecruit}, mainly hire freelancers and outsource work[8] to cut expenses. In addition to this, there exists an ever-growing stigma on otaku culture that is mainly caused by the industry's recurring themes in storytelling and character design~\cite{otakuObsessed, otakuMeaning, otakuWhyHated}.

In line with this, it is of great interest to solve these three aspects of the anime industry: mainly, \textbf{the inefficiencies of production ~\cite{animeProductionFull, animeProductionDb, animeProductionSummary, animeProductionJapan}}, \textbf{the work environment with low minimum wage and gruelling work hours~\cite{animeIndustryHowLowSalary}}, and \textbf{the stigma on otaku culture discussed previously}.

In this paper, I present a full pipeline on a combination of techniques taken from both western and Japanese studios and creators applied on the production of a short film, its pros and cons, and the potential research and applications of further improvement of the pipeline using both unconventional and classical technologies that include but are not limited to motion capture, computer science concepts, and machine learning. The main idea of the pipeline can be broken down into writing, storyboard direction, animation, and post-processing. So far, this study has been successfully applied to the first three steps: writing, storyboard direction, and animation. The short film was completed as a draft within the deadline, but post-processing has not been applied.

The film's storyline is created as a concept based on the both American band Paramore's song Brighter and the my own experiences, which is then drafted into character designs, object \& costume list, and a mockup storyboard. The character designs are then converted into rough 3D models using Blender, and are used as animation reference for all the movements after converting the mockup storyboard into a full, finalised video with dynamic camera view angles that perfectly follow the song's transitions. These movements, camera angles, and transitions are then reviewed and re-tweaked easily due to automation of the storyboarding process with acting and motion capture. The finalised video is then treated as a full step-by-step guide to simply follow for animation and post-processing.

The research and production has been created with the restriction of having a full-time job, thereby limiting the amount of work to roughly 4 to 6 hours on weekdays and 8 to 12 hours on weekends. A deadline of August 31 has been set to prevent open-endedness and unlimited leeway for rework and editing. In addition, I do not have formal education in film and animation, and all parts of the process are simply created from experiences in acting in indie films \& theater, first principles, computer science and basic machine learning background, and references on the current animation industries' methods. My current drawing and animation abilities are less than average, with only a single anime short film posted online in 2017 as experience.